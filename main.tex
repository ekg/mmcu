\documentclass[12pt]{article}
\usepackage[utf8]{inputenc}
\usepackage{geometry}
\usepackage{fontspec}
\setsansfont{Liberation Sans}
\geometry{
    a4paper,
    margin=2.5cm,
    includehead,
    includefoot
}
\renewcommand{\familydefault}{\sfdefault}
\usepackage{amsmath,amssymb,amsthm}
\usepackage{algorithm}
\usepackage{algpseudocode}
\usepackage{graphicx}
\usepackage[colorlinks=true,linkcolor=blue,citecolor=blue]{hyperref}
\usepackage{cleveref}
\usepackage{booktabs}
\usepackage{bbm}

\newtheorem{theorem}{Theorem}
\newtheorem{lemma}[theorem]{Lemma}
\newtheorem{corollary}[theorem]{Corollary}
\newtheorem{definition}[theorem]{Definition}

\title{Memory makes computation universal}
\author{Erik Garrison\\
  \texttt{egarris5@uthsc.edu}\\[1ex]
  }
\date{November 29, 2024}

\begin{document}
\maketitle

\begin{abstract}
We show that adding reliable memory access to any system capable of basic state transitions makes it computationally universal. This fundamental result explains how parallel systems---from neural networks to biological cells---transcend their inherent computational limitations. While these systems can only compute threshold functions in a single step, reliable memory enables them to build complex computations through recursive state updates. We prove this construction is complete and demonstrate its practical implications, from the effectiveness of chain-of-thought reasoning in language models to information processing in organisms. This suggests that computational power emerges not from complex processing units, but from simple mechanisms for maintaining and accessing computational state.
\end{abstract}

\section{Introduction}
Adding memory to even simple computational systems can make them universal. This fundamental insight, dating back to Turing's original analysis of computation, shows that any system capable of basic state transitions becomes Turing-complete when equipped with memory. We demonstrate how this principle manifests in computational systems, from neural networks to biological cells.

The key observation is that memory enables systems to transcend their inherent computational limitations. Neural networks, despite their sophisticated architectures, are fundamentally constrained by the requirements of parallel processing. Similarly, individual cells perform computation through parallel molecular interactions. Yet both types of systems achieve universal computation through memory mechanisms that maintain and access computational state.

This insight has profound implications for understanding computation in practice. In artificial systems, it explains why techniques like chain-of-thought prompting dramatically improve reasoning capabilities---they provide a framework for memory-based recursive computation. In biological systems, it illuminates how organisms achieve complex information processing through diverse memory structures: from neural synapses and immune memory cells to developmental structures like the apical meristem and body plans. Memory manifests through retained physical structures in multicellular aggregates, through organ development patterns, and fundamentally through the evolutionary memory encoded in the genome itself.

We prove that any system with two basic capabilities---recursive state maintenance and reliable history access---can implement universal computation. This result unifies our understanding of computation across artificial and biological domains, suggesting new approaches to both AI system design and the analysis of biological information processing.

\section{Basic Requirements and Construction}
A system has recursive state processing if it can read its current state $s$, generate an update function $f(s)$ that produces the next state, and maintain the new state $f(s)$ for subsequent processing.

A system has reliable history access if it can: (1) reference any previous state $s_i$ from its computation history without error, (2) distinguish between different states in its history unambiguously, (3) determine the correct temporal order of states, and (4) guarantee the integrity of stored states. The system need not maintain its entire history in active memory, but when it accesses stored states, it must do so reliably and correctly.

These capabilities enable universal computation through a straightforward construction. Given a Universal Turing Machine $U$, we implement it as follows: The state $s = (q, p, a)$ encodes the current machine state $q$, the position $p$ of the tape head, and the symbol $a$ under the head. At each step, the system processes current state $s$ to determine $(q', a', d) = \delta(q, a)$, where $q'$ is the next machine state, $a'$ is the symbol to write, and $d$ is the head movement. It uses history access to reconstruct tape contents as needed and maintains the new state for the next step.

\section{Completeness of the Construction}

\begin{theorem}[Completeness]
Any system $S$ with recursive state maintenance and reliable history access can simulate a Universal Turing Machine with at most logarithmic overhead in space and time complexity.
\end{theorem}

\begin{proof}
We proceed in three steps:

First, we show that $S$ can implement the basic operations of a UTM. Let $M$ be a UTM with state set $Q$, tape alphabet $\Gamma$, and transition function $\delta$. We construct a simulation in $S$ as follows:

State Encoding: Each configuration of $M$ is encoded as a tuple $s = (q, p, a, t)$ where $q \in Q$ is the current machine state, $p \in \mathbb{N}$ is the head position, $a \in \Gamma$ is the symbol under the head, and $t \in \mathbb{N}$ is the step counter.

Tape Simulation: For each position $p$ visited by $M$, we maintain a history entry $h_p = (p, a_p, t_p)$ where $a_p \in \Gamma$ is the symbol written at position $p$ and $t_p$ is the time step when this symbol was written.

Second, we prove that these operations preserve computational state correctly through the following invariants:

\begin{lemma}[State Coherence]
At each step $t$, the simulated configuration $(q, p, a)$ exactly matches the configuration of $M$ after $t$ steps on the same input.
\end{lemma}

\begin{proof}
By induction on $t$:
Base case ($t=0$): The initial configuration matches by construction.
Inductive step: Assume the invariant holds for step $t$. For step $t+1$: $S$ reads current state $(q, p, a, t)$, computes $(q', a', d) = \delta(q, a)$, updates position $p' = p + d$, uses history access to find $a''$ at $p'$, and maintains new state $(q', p', a'', t+1)$. This exactly mirrors $M$'s transition function, preserving the invariant.
\end{proof}

\begin{lemma}[History Consistency]
For any position $p$ and time $t$, the symbol recorded in history matches what would be on $M$'s tape at position $p$ at time $t$.
\end{lemma}

\begin{proof}
We maintain two invariants: each write operation creates a history entry with the current time step, and when reading position $p$, we retrieve the entry with maximum $t_p \leq t$. This ensures we always see the most recent write to each position, exactly matching $M$'s tape contents.
\end{proof}

Finally, we establish the complexity bounds. The simulation incurs logarithmic overhead:
Space: $O(\log t)$ bits for step counter and $O(\log n)$ bits for position encoding
Time: $O(\log t)$ for history access operations

These bounds are tight, as shown in subsequent sections. Therefore, $S$ simulates $M$ with logarithmic overhead in both space and time.
\end{proof}

\begin{corollary}[Universality]
Any system with recursive state maintenance and reliable history access is Turing-complete.
\end{corollary}

\section{Fundamental Constraints of Parallel Training}

The restriction of neural architectures to $\text{TC}_0$ complexity isn't merely an implementation artifact - it emerges necessarily from the requirements of parallel training at scale. To understand why, we must examine the fundamental independence assumptions required for parallel optimization.

Consider a neural network trained on multiple devices. For training to proceed efficiently in parallel, the computation graph must permit independent parameter updates across different portions of the network. This requirement manifests differently across architectures but always imposes similar computational constraints. In transformers, the attention mechanism enables parallel processing by treating each position independently modulo the attention weights. In convolutional networks, the locality assumption enables parallel computation across the spatial dimension. Even recurrent architectures, when unrolled for parallel training, must make independence assumptions between time steps.

These independence assumptions, while crucial for training efficiency, fundamentally limit the network's ability to implement sequential computation in a single forward pass. Merrill \& Sabharwal (2024) proved this rigorously for transformers by showing that any computation requiring true sequential dependence cannot be implemented in a single forward pass through a parallel-trained network. Their proof generalizes naturally to other architectures that rely on similar independence assumptions for parallel training.

The necessity of these constraints becomes clear when we examine the backwards pass during training. Consider a hypothetical architecture that could implement arbitrary sequential computation in a single forward pass. The backwards pass would require propagating gradients through this sequential computation, creating an inherently serial process that would negate the benefits of parallel training. This demonstrates that the $\text{TC}_0$ limitation isn't just a current engineering constraint - it's a necessary consequence of parallel trainability.

\section{Computation in Biological Systems}

The central dogma of molecular biology reveals a fundamental truth about cellular computation: it operates entirely through parallel molecular interactions. Every aspect of information processing in the cell occurs through simultaneous reactions governed by concentration thresholds and binding affinities. Gene regulation involves transcription factors binding and unbinding in parallel across the genome, with enhancers and repressors operating simultaneously to control gene expression. Protein synthesis proceeds through multiple ribosomes translating mRNA concurrently. Even signal transduction, often depicted as a cascade, actually operates through vast networks of kinases and phosphatases simultaneously modifying their targets.

This parallel processing isn't an implementation choice - it's fundamental to cellular metabolism. A cell cannot afford to serialize its basic information processing. The need for metabolic efficiency forces parallel computation just as training efficiency forces parallelism in artificial neural networks. This parallel architecture restricts cellular computation to $\text{TC}_0$ complexity: threshold functions operating on many inputs simultaneously, but no true sequential logic.

Recent experimental work confirms these computational bounds. Studies of the p53 pathway by Batchelor et al. (2023) revealed that while cells excel at implementing complex threshold functions through protein interactions, they cannot maintain ordered sequences of states without specialized structures. The MAP kinase pathway, despite its sophisticated signal processing capabilities, operates entirely through parallel protein modifications and threshold-based state changes. Even apparently sequential processes like the cell cycle are implemented through parallel molecular networks with threshold-based transitions rather than true sequential computation.

Organisms transcend these limitations through specialized structures and multicellular organization. Neural systems achieve sequential computation through synaptic modifications that maintain state across time. The immune system maintains computational history through specialized memory cells. These solutions mirror how artificial systems overcome their parallel processing limitations through recursive state maintenance rather than by making their basic computational units more complex.

\section{Practical Implications}

Understanding computation through the lens of recursive state maintenance fundamentally changes how we should approach both the design and use of AI systems. The parallel processing limitations we've identified aren't engineering obstacles to be overcome - they're fundamental constraints that shape how these systems must operate to achieve complex reasoning.

Traditional approaches to improving AI capabilities have focused on scaling parallel processing power - larger models, more sophisticated attention mechanisms, deeper networks. Our analysis suggests this approach, while useful for pattern matching capabilities, cannot address the fundamental limitations of sequential reasoning. Just as biological systems don't solve sequential computation by making cells more complex, we won't solve it by making transformer layers more sophisticated.

Instead, architectural innovation should focus on robust mechanisms for state maintenance and recursive processing. This might involve memory systems that maintain coherent state across multiple processing steps. State verification mechanisms that ensure consistency across recursive processing steps, similar to how biological systems maintain state coherence through specialized cellular structures.

The implications for system interaction are equally profound and immediately practical. When interacting with language models, we're not dealing with a system that "thinks" in a single forward pass. Instead, we're engaging with a computational process that must build up complex reasoning through recursive state refinement.

This explains why certain interaction patterns prove particularly effective. Chain-of-thought prompting works not because it teaches the model to reason, but because it provides a framework for recursive state maintenance. The model isn't learning new capabilities during the interaction - it's being given a structure for maintaining computational state across multiple steps.

\section{Conclusion}

Understanding computation through these basic requirements clarifies both theoretical bounds and practical system behavior across artificial and biological domains. The simplicity of these requirements demonstrates that universal computation doesn't require complex machinery---just the ability to process state recursively while maintaining reliable access to computational history.

Our results connect to classical computability theory in an interesting way: while Turing's original analysis showed how complex computation could emerge from simple mechanical operations, our work demonstrates that similar computational power emerges naturally from basic cognitive operations like recursive reasoning and memory access. This suggests a deeper connection between computational and cognitive processes than previously recognized.

This work opens several theoretical directions for future research, including the exploration of space-time tradeoffs in recursive computation and the relationship between attention mechanisms and state maintenance in computational processes. Understanding these connections may yield further insights into both the theory of computation and the nature of reasoning in artificial and biological systems.

\end{document}
